\documentclass[12pt,a4paper,svgnames]{article}
\usepackage[utf8]{inputenc}
\usepackage[T1]{fontenc}
\usepackage{lmodern}
\usepackage{amsmath}
\usepackage{amsfonts}
\usepackage{amssymb}
\usepackage{graphicx}
\usepackage[spanish,es-nodecimaldot,es-lcroman,es-tabla,es-noshorthands]{babel}
\usepackage[left=3cm,right=2cm, bottom=4cm]{geometry}
\usepackage{subfigure}
\usepackage{tikz}
\usepackage{xstring}
\usepackage{./tikz-uml}

%\setsansfont[Ligatures=TeX]{texgyreadventor}
%\setmainfont[Ligatures=TeX]{texgyrepagella}

\input{portada}
\newcommand{\anadirproductoadministrador}{
\begin{tikzpicture}
\tikzumlset{fill state=white}
\umlstateinitial[name=initial]
\begin{umlstate}[x=0, y=-3,name=datos]{Iniciar sesión}
\end{umlstate}

\umlstatedecision[y=-5, name=permiso] 

\begin{umlstate}[x=0, y=-8,name=comprobar]{Añadir nuevo producto}
\end{umlstate}

\begin{umlstate}[y=-11,name=sesion]{Rellenar datos del producto}
\end{umlstate}

\begin{umlstate}[y=-14,name=actualizar]{Confirmar nuevo producto}
\end{umlstate}

\umlstatefinal[y=-16, name=final]

\umltrans{initial}{datos}
\umltrans{datos}{permiso}
\umltrans[arg1=correcto]{permiso}{comprobar}
\umltrans{comprobar}{sesion}
\umltrans{sesion}{actualizar}
\umltrans{actualizar}{final}
\umlHVHtrans[arg1=incorrecto, arm2=-4]{permiso}{datos}

\end{tikzpicture}
}

\newcommand{\comprarproducto}{
	\begin{tikzpicture}
	\tikzumlset{fill state=white}
	\umlstateinitial[name=initial]
	\begin{umlstate}[x=0, y=-3,name=datos]{Buscar producto}
	\end{umlstate}
	
	\umlstatedecision[y=-5, name=permiso] 
	
	\begin{umlstate}[x=0, y=-8,name=comprobar]{Añadir prodcuto al carrito}
	\end{umlstate}
	
	\begin{umlstate}[y=-11,name=sesion]{Ver cesta}
	\end{umlstate}
	
	\begin{umlstate}[y=-14,name=actualizar]{Realizar pedido}
	\end{umlstate}
	
	\umlstatefinal[y=-16, name=final]
	
	\umltrans{initial}{datos}
	\umltrans{datos}{permiso}
	\umltrans[arg1=encontrado]{permiso}{comprobar}
	\umltrans{comprobar}{sesion}
	\umltrans{sesion}{actualizar}
	\umltrans{actualizar}{final}
	\umlHVHtrans[arg1=no encontrado, arm2=-4]{permiso}{final}
	
	\end{tikzpicture}
}

\newcommand{\casosdeuso}{
	\begin{tikzpicture}
	\tikzumlset{fill usecase=white}
	\tikzumlset{fill note=gray!30}
	\tikzumlset{font=\tiny}
	
	\umlactor[x=0, y=3]{USUARIO}
	\umlactor[x=0, y=-2]{ADMIN}
	
	\umlusecase[x=3, y=4, name=buscarproductos]{Buscar productos}
	\umlusecase[x=3, y=2, name=realizarpedido]{Realizar pedido}
	\umlusecase[x=8, y=3, name=anadirproductocarro]{Añadir producto producto al carro}
	\umlusecase[x=8, y=2, name=vercarro]{Ver carro}
	
	\umlassoc{USUARIO}{realizarpedido}
	\umlassoc{USUARIO}{buscarproductos}
	
	
	\umlinclude{realizarpedido}{anadirproductocarro}
	\umlinclude{realizarpedido}{vercarro}
	
	
	\umlusecase[x=3, y=-1, name=anadirproducto]{Añadir productos}
	\umlusecase[x=3, y=-2, name=verproductos]{Ver productos}
	\umlusecase[x=3, y=-3, name=verpedido]{Ver pedido}

	\umlusecase[x=7, y=-2, name=modificarproducto]{Modificar producto}
	\umlusecase[x=7, y=-3, name=borrarproductos]{Borrar productos}
	\umlusecase[x=7, y=-4, name=borrarpedido]{Borrar pedido}
	\umlusecase[x=7, y=-5, name=actualizarestadopedido]{Actualizar estado pedido}
	
	\umlusecase[x=11, y=-2, name=iniciarsesion]{Iniciar sesión}
	
	\umlassoc{ADMIN}{anadirproducto}
	\umlassoc{ADMIN}{verproductos}
	\umlassoc{ADMIN}{verpedido}
	
	
	
	\umlinclude{anadirproducto}{iniciarsesion}
	\umlextend{modificarproducto}{verproductos}
	\umlextend{borrarproductos}{verproductos}
	\umlextend{borrarpedido}{verpedido}
	\umlextend{actualizarestadopedido}{verpedido}
	\umlextend{realizarpedido}{vercarro}
	
	\umlinclude{modificarproducto}{iniciarsesion}
	\umlinclude{borrarproductos}{iniciarsesion}
	\umlinclude{borrarpedido}{iniciarsesion}
	\umlinclude{actualizarestadopedido}{iniciarsesion}
	
	
	\end{tikzpicture}
}

\newcommand{\clases}{
	
\begin{tikzpicture}
	

\umlclass[x=0, y=0]{ProductWithCuantity}{
}{
}

\umlclass[x=5, y=0]{Cart}{
}{
}

\umlclass[x=0, y=-5]{Product}{
}{
}

\umlclass[x=5, y=-5]{Order}{
}{
}

% Relaciones

\umlcompo[mult2=1..n]{Cart}{ProductWithCuantity}

\umlcompo[mult2=1..n]{Order}{Cart}

\umlassoc[mult2=1, mult1=1..n]{ProductWithCuantity}{Product}


\end{tikzpicture}

}


\newcommand*{\autores}{
\begin{tabular}{r l}
GII+GIS: & Germán Alonso Azcutia \\
GIS+MAT: & José Ignacio Escribano Pablos
\end{tabular}
}

\begin{document}

\pagenumbering{alph}
\setcounter{page}{1}

\portada{Práctica 1}{Diseño de Aplicaciones Web}{Implementación de una tienda virtual}{\autores}{Móstoles}

\listoffigures
\thispagestyle{empty}
\newpage

\tableofcontents
\thispagestyle{empty}
\newpage


\pagenumbering{arabic}
\setcounter{page}{1}


\section{Introducción}
Para la realización de la práctica vamos a usar el patrón de arquitectura de software \\
\texttt{Modelo-Vista-Controlador(MVC)}, que como su propio nombre indica, define la organización independiente del Modelo (información y lógica de negocio), la Vista (interfaz con el usuario) y el Controlador (intermediario entre modelo y vista).\\

\begin{figure}[htbp]
\centering
\subfigure{\includegraphics[width=65mm]{./imagenes/mvc}}
\caption{Patrón de arquitectura Modelo Vista Controlador (MVC)}
\end{figure}

Para llevar a cabo la explicación de la responsabilidad de cada clase, vamos a dividir dicha explicación en las tres partes del patrón.\\

Pero antes, veamos los diagramas \texttt{UML} de nuestra aplicación web.

\clearpage

\section{Diagramas UML}
\subsection{Diagrama de casos de uso}
 
El diagrama de casos de uso de la Figura~\ref{fig:casosdeuso} muestra que la aplicación tiene dos roles distintos dentro de la aplicación: usuario y administrador. El primero de ellos puede ver productos, añadirlos al carrito, ver su carro, y el segundo puede añadir productos al catálogo de la tienda, modificarlos, ver los pedidos, así como todas las acciones derivadas de las anteriores.\\

Esto supone que hay que distinguir casos distintos casos, ya sea para un usuario o para el administrador.  
\begin{figure}[htbp]
	\centering
	\casosdeuso
	\caption{Diagrama de casos de uso de la aplicación}
	\label{fig:casosdeuso}
\end{figure}

\clearpage

\subsection{Diagrama de clases}

El diagrama de clases muestra todas las clases que manejarán la lógica de la aplicación. Estas clases son ProductWithCuantity, Cart, Product y Order. La descripción de cada clase se puede ver en la Sección~\ref{sec:descripcion}.

\begin{figure}[htbp] 
	\centering
	\clases
	\caption{Diagrama de clases de la aplicación}
\end{figure}

\subsection{Diagramas de actividad}

Como hemos comentado anteriormente, debemos distinguir tareas para un usuario normal y el administrador. Esto se refleja en los diagramas de actividad: en el caso de un usuario normal no es necesario iniciar sesión (en el enunciado de la práctica se especifica que no hay usuarios, salvo el administrador), mientras que en el adminstrador sí que es necesario.\\

Un ejemplo de lo anterior se puede ver en las Figuras~\ref{fig:actividad1}~y~\ref{fig:actividad2} donde no es necesario iniciar sesión para comprar un producto de la tienda, y, por el contrario, sí que lo es para añadir un producto al catálogo de la tienda.  

\begin{figure}[htbp] 
	\centering
	\comprarproducto
	\caption{Diagrama de secuencia de comprar producto}
	\label{fig:actividad1}
\end{figure}

\begin{figure}[htbp]
	\centering
	\anadirproductoadministrador
	\caption{Diagrama de secuencia de añadir producto como administrador}
	\label{fig:actividad2}
\end{figure}

\clearpage

\section{Descripción de las clases}\label{sec:descripcion}
\subsection{Modelo}
\subsubsection{Clase Product}
Esta clase define el objeto producto y las propiedades y métodos de los que dispondrán los productos de nuestra web.\\
\subsubsection{Clase ProductWithCuantity}
La clase \texttt{ProductWithCuantity} define un producto que estará en el carro de compra y la cantidad de dicho producto almacenada en el carro. Por tanto, se compone de un producto y su cantidad.\\
\subsubsection{Clase Cart}
Esta clase define el carro de compra de cada usuario. Está compuesto por un ArrayList de \texttt{ProductWithCuantity} (que serán el total de productos en la cesta) y el precio total del carro.\\
\subsubsection{Clase Order}
La clase \texttt{Order} define el pedido del usuario, es decir, la confirmación de querer comprar su carro de compra. Por ello, esta clase contiene un atributo carro, además del estado del pedido (pendiente o preparado) y los datos correspondientes del usuario que ha realizado el pedido.\\
\subsubsection{Interfaz ProductRepository}
Es una clase que extiende de CrudRepository, interfaz genérica que ofrece operaciones tipo CRUD (Create-Read-Update-Delete) para acceder a los datos de la base de datos. Nosotros lo utilizaremos en esta clase para realizar consultas sobre los productos en la BD.\\
\subsubsection{Interfaz OrdertRepository}
Al igual que \texttt{ProductRepository}, esta clase extiende de CrudRepository, pero esta vez la clase nos servirá para realizar consultas a la BD sobre los pedidos.\\
\subsection{Controlador}
\subsubsection{Clase ProductController}
Esta clase es el controlador que se encarga de gestión todas las acciones relacionadas con los productos del catálogo. Ejemplo de esto puede ser mostrar información de productos, listar todos los productos, añadir productos al carrito de la compra, buscar (tanto por categorías, por nombre o precio). \\

Este controlador recoge las peticiones del repositorio y lleva la información proporcionada por el repositorio hasta la vista correspondiente a la acción requerida.

\subsubsection{Clase OrderController}
Análogo al anterior, pero gestionando los pedidos. Realiza acciones tales como mostrar pedidos, confirmarlos o pedir los datos (nombre y apellidos) del pedido.

\subsubsection{Clase AdminController}
Este controlador es el encargado de gestionar las peticiones desde la sección de administraddor como añadir nuevo producto, borrar productos, editar productos, cambiar estados de los pedidos, etc.

\subsubsection{Clase StartController}
Esta clase es un controlador que implementa la interfaz \texttt{CommandLineRunner}, que permite ejecutar algunos comandos al iniciar la aplicación. Se encarga de introducir algunos productos y pedidos de prueba que sirvan para probar el correcto funcionamiento de la aplicación. En la versión de producción, este controlador se eliminaría.
\subsection{Vista}
La interfaz de nuestra aplicación se divide en las siguientes vistas:
\texttt{\textbackslash index.html, \textbackslash product.html, \textbackslash cart.html, \textbackslash admin.html, \textbackslash new.html, \textbackslash edit.html, \textbackslash order.html y \textbackslash orders.html} que se explicarán en la sección de \texttt{Recorrido por la aplicación}


\clearpage

\section{Recorrido por la aplicación}

\subsection{index.html}
Esta es la vista principal de nuestra aplicación web. Como se puede observar en la Figura~\ref{fig:index_sin_admin}, en la parte superior los usuarios puedes buscar productos por categoría y por nombre, y el administrador podrá loguearse con el menú desplegable. Estas opciones también estarán disponibles en las demás vistas. Cuando hemos iniciado sesión, se cambia este desplegable con un cerrar sesión.

\begin{figure}[htbp]
\centering
\includegraphics[width=1\linewidth]{imagenes/index_con_admin}
\caption{Vista index.html}
\label{fig:index_sin_admin}
\end{figure}

En el cuerpo de la vista encontramos en la parte izquierda el listado de los productos de la web, y si han realizado una búsqueda, aparecerán los productos que concuerden con dicha búsqueda. 
Para cada producto aparecen 2 botones, el botón \texttt{Carro} que añadirá el producto a la cesta y el botón \texttt{Más información}, que nos llevará a la vista \texttt{/product.html} para ver en detalle dicho producto.\\

Por último, en la parte derecha tenemos la opción de filtrar los productos por precio y el carro de compra, donde se pueden ver los productos añadidos y un botón para acceder a la vista \texttt{/cart}, que contendrá el carro de compra (Figura~\ref{fig:moreInformation}).

\clearpage

\subsection{product.html}
En esta vista (Figura~\ref{fig:moreInformation}) se puede observar el producto seleccionado con la descripción completa y la opción de añadirlo al carro.

\begin{figure}[htbp]
	\centering
	\includegraphics[width=1\linewidth]{imagenes/moreInformation}
	\caption{Vista product.html}
	\label{fig:moreInformation}
\end{figure}

\subsection{cart.html}
En esta vista tenemos (Figura~\ref{fig:cart}) todos los productos que han sido añadidos a la cesta de compra. En ella podemos aumentar la cantidad de un producto o, si lo deseamos, borrar un producto de la cesta. Cuando ya esté lista la cesta, se podrá pulsar el botón de hacer pedido, que nos llevará a la vista \texttt{order.html} con un mensaje confirmando el pedido.


\begin{figure}[htbp]
	\centering
	\includegraphics[width=1\linewidth]{imagenes/cart}
	\caption{Vista cart.html}
	\label{fig:cart}
\end{figure}


\subsection{order.html}
Esta vista (Figura~\ref{fig:order}) como se puede observar, podemos ver el carro de compra que hemos hecho y tenemos un pequeño formulario para rellenar antes de proceder a confirmar el pedido (botón realizar pedido). Al realizar el pedido, se volverá a la vista \texttt{/index.html}

\begin{figure}[htbp]
	\centering
	\includegraphics[width=1\linewidth]{imagenes/order}
	\caption{Vista order.html}
	\label{fig:order}
\end{figure}

\clearpage

\subsection{admin.html}
Esta vista (Figura~\ref{fig:admin}) es la pantalla principal del administrador, en ella aparecen los distintos productos de la web, donde tenemos las opciones de añadir nuevos productos, borrar o editar productos existentes, filtrar los productos o ver los pedidos.

\begin{figure}[htbp]
	\centering
	\includegraphics[width=1\linewidth]{imagenes/admin}
	\caption{Vista admin.html}
	\label{fig:admin}
\end{figure}

\subsection{new.html}
En esta vista (Figura~\ref{fig:new}) tenemos un formulario que nos permitirá añadir nuevos productos a la tienda.

\begin{figure}[htbp]
	\centering
	\includegraphics[width=1\linewidth]{imagenes/new}
	\caption{Vista new.html}
	\label{fig:new}
\end{figure}

\subsection{edit.html}
En esta vista (Figura~\ref{fig:edit}) tenemos otro formulario que nos permitirá editar los productos de la tienda.

\begin{figure}[htbp]
	\centering
	\includegraphics[width=1\linewidth]{imagenes/edit}
	\caption{Vista edit.html}
	\label{fig:edit}
\end{figure}

\clearpage

\subsection{orders.html}
En \texttt{orders.html} (Figura~\ref{fig:orders}) se mostrarán los distintos pedidos de la tienda donde tendremos la opción de cambiar su estado (pendiente o preparado) (Figura~\ref{fig:orders_actualizar_estado}). Además, se podrán filtrar los pedidos mediante su estado.

\begin{figure}[htbp]
	\centering
	\includegraphics[width=1\linewidth]{imagenes/orders}
	\caption{Vista orders.html}
	\label{fig:orders}
\end{figure}

\begin{figure}[htbp]
	\centering
	\includegraphics[width=1\linewidth]{imagenes/orders_actualizar_estado}
	\caption{Vista orders.html (actualizar estado)}
	\label{fig:orders_actualizar_estado}
\end{figure}



\end{document}
